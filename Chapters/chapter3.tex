%!TEX root = ../template.tex
%%%%%%%%%%%%%%%%%%%%%%%%%%%%%%%%%%%%%%%%%%%%%%%%%%%%%%%%%%%%%%%%%%%%
%% chapter2.tex
%% NOVA thesis document file
%%
%% Chapter with the template manual
%%%%%%%%%%%%%%%%%%%%%%%%%%%%%%%%%%%%%%%%%%%%%%%%%%%%%%%%%%%%%%%%%%%%
\chapter{Related Work}
\label{cha:related_work}
On this thesis, it is intended to apply Human-computer interaction (HCI), Data Visualization and Visual Querying concepts, techniques and technologies to improve a Visual Querying Feature of the OutSystems Low-code Development Platform. Thus, in this chapter, will be presented the results of a study that analysed what is the Low-code development platform background and its actual situation, as well as what are the techniques and technologies which already exist, including some comparison between them. Finally, will be enumerated what products, technologies and tools exist on other commercial applications which can be related with the topics of this thesis.

\section{Visual Query Formulation}
\label{sec:visual_query_formulation}

\section{Data Visualization}
\label{sec:data_visualization}

\section{Data User Exprerience and Expressiveness}
\label{sec:data_user_experience_and_expressiveness}

\section{Discussion}
\label{sec:discussion}