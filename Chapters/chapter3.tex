%!TEX root = ../template.tex
%%%%%%%%%%%%%%%%%%%%%%%%%%%%%%%%%%%%%%%%%%%%%%%%%%%%%%%%%%%%%%%%%%%%
%% chapter2.tex
%% NOVA thesis document file
%%
%% Chapter with the template manual
%%%%%%%%%%%%%%%%%%%%%%%%%%%%%%%%%%%%%%%%%%%%%%%%%%%%%%%%%%%%%%%%%%%%
\chapter{Related Work}
\label{cha:related_work}
In this chapter, will be described and compared a set of technologies and techniques, which could be useful to explore solutions, and understand different points of view to manage problems, regarding the project scope. As the project is focused on the improvement of an interface that combines the query design with the query output viewer, some relevant technologies, regarding the former will be compared, and, in a different section, the related work that exists for the latter shall be presented. Moreover, since the users are a fundamental piece of this study, will be presented some information about how is its relationship with data and what are its expectations to these query systems.

\section{Visual Query Formulation}
\label{sec:visual_query_formulation}

The first relevant topic to analyse is the visual representation approaches, which can be applied to design an user interface capable to build queries through a visual language. Catarci \textit{et al.} \cite{visualQuerySystemsForDatabases_aSurvey} presented an interesting classification according to the visual formalism which the interface is based on:

\begin{itemize}
    \item \textbf{Form-based}: based on forms, which can be seen as a rectangular grid of other components (subforms, groups of cells, a combination of cells, etc.) that group objects in a named collection regarding its structure. Forms and tables are similar,  but contrary to the tables, forms allow nesting. Thus, forms can be seen as a generalization of tables. In this approach, the relationships can be represented among cells, cells subsets, or even the overall set, providing to the user three information levels;
    \item \textbf{Diagram-based}: usage of graphical representations, such as graphs, charts and diagrams to better transmit the relationships among data. The aim is to use visual representations to help the understanding of the relationships between concepts which are represented by textual labels;
    \item \textbf{Icon-based}: as the opposite of the diagram-based, this type of interfaces tries to facilitate the understanding of the concepts instead of relationships. So, are used Icons, which are visual segmented objects to transmit a message or information, using analogies and metaphors with the real-world objects, or even conventions that are used to express no tangible objects, as computer processes;
    \item \textbf{Hybrid}: these approaches can combine the previous visual formalisms in order to select the best combination of advantages to the application usage domain.
\end{itemize}

Following, it is essential to specify what are the visual query language requisites, to understand and compare the different parts of the systems. Accordingly, Table \ref{tab:query_language_requisites} presents the query creation required specifications, comparing them with the respective indication in SQL.

\begin{table}[ht]
	\caption{Query Language Requisites}
	\label{tab:query_language_requisites}
\centering
\begin{tabular}{|
    >{\columncolor[HTML]{EFEFEF}}l |l|l|}
    \hline
    \cellcolor[HTML]{C0C0C0}{\color[HTML]{333333} \textbf{Specification}} & \cellcolor[HTML]{C0C0C0}{\color[HTML]{333333} \textbf{Description}}                                                                                               & \cellcolor[HTML]{C0C0C0}{\color[HTML]{333333} \textbf{SQL Indication}}                                                                   \\ \hline
    \textbf{Data Source}                                                  & \begin{tabular}[c]{@{}l@{}}Entities and attributes which \\ will be presented in the query\end{tabular}                                                           & \begin{tabular}[c]{@{}l@{}}Using SELECT and \\ FROM statements\end{tabular}                                                              \\ \hline
    \textbf{Merge Type}                                                   & \begin{tabular}[c]{@{}l@{}}Define how will be merged \\ attributes of different entities\end{tabular}                                                             & Using JOIN clauses                                                                                                                       \\ \hline
    \textbf{Filtering Criteria}                                           & \begin{tabular}[c]{@{}l@{}}Criteria that can be used to \\ filter records, presenting in the \\ result only those that fulfil a \\ set of conditions\end{tabular} & \begin{tabular}[c]{@{}l@{}}Using WHERE \\ clause\end{tabular}                                                                            \\ \hline
    \textbf{Sorting Criteria}                                             & \begin{tabular}[c]{@{}l@{}}Define what are the criteria to \\ sort the records of the result\end{tabular}                                                         & \begin{tabular}[c]{@{}l@{}}Using ORDER BY \\ clause\end{tabular}                                                                         \\ \hline
    \textbf{Aggregation Functons}                                         & \begin{tabular}[c]{@{}l@{}}Group a set of records by \\ comparison or using mathematical \\ functions\end{tabular}                                                & \begin{tabular}[c]{@{}l@{}}Using  GROUP BY \\ statements or SQL \\ functions, such as \\ MIN, MAX, \\ COUNT, AVG and \\ SUM\end{tabular} \\ \hline
    \textbf{Calculated Attributes}                                        & \begin{tabular}[c]{@{}l@{}}Attributes added, based on \\ existing ones\end{tabular}                                                                               & \begin{tabular}[c]{@{}l@{}}Using SELECT \\ statement\end{tabular}                                                                        \\ \hline
    \textbf{Distinct Values}                                              & \begin{tabular}[c]{@{}l@{}}If only different values will be \\ considered in the result \\ (removing duplicated values)\end{tabular}                              & \begin{tabular}[c]{@{}l@{}}Using SELECT \\ DISTINCT statement\end{tabular}                                                               \\ \hline
    \textbf{Unions}                                                       & \begin{tabular}[c]{@{}l@{}}Combine the result of two \\ different queries\end{tabular}                                                                            & \begin{tabular}[c]{@{}l@{}}Using UNION \\ operator\end{tabular}                                                                          \\ \hline
    \textbf{Subqueries}                                                   & \begin{tabular}[c]{@{}l@{}}Defining a query that uses \\ other queries, for example, \\ to filter the result\end{tabular}                                         & \begin{tabular}[c]{@{}l@{}}Nesting SELECT \\ statements\end{tabular}                                                                     \\ \hline
    \end{tabular}
\end{table}

Nevertheless, there are two relevant aspects, according to the last requisites presented: the interaction process to indicate the query specifications, the interface feedback about what the current query. Both are fundamental since a good visual query language aims to simplify not only, the design process but also, the query readability, promoting a faster and easier recognition of what are the desired data.

\bigskip

\textbf{Data Source:}

Chartio \cite{chartio} has two components to query databases visually: using the Data Explorer \cite{chartioDataExplorer} or using the new Visual SQL \cite{chartioVisualSQL}. Regarding the data source specification, these two systems use different strategies to select and present the entities and attributes related to the query. In the Data Explorer, there is a list of tables in a fixed, scrollable and searchable tree view, where the user can expand each one and choose the desired attributes. Also, above this list of collapsible items, there is a search component that can be used to select the desired attributes. This system divides the attributes into two different types: Measures and Dimensions. Usually, measure refers to quantitative data and dimensions to categorical data. So, to insert the attributes in the query, users can drag and drop the required attributes to the form-based interface that contains the Measures, Dimensions and Filters of the query (Figure \ref{fig:chartioDataExplorer}) \cite{chartioDataExplorer}. 

On the other hand, the new component of Chartio, the Visual SQL provides a different interface to select the data sources. Contrary to the previous approach, there is any fixed list in any part of the window to choose the attributes required to the query. In this way, there is only a search text component that is activated when the user clicks on “add column” action. After this, is presented over a new interface component that has a list similar to the referred above where is possible to view a preview of some data entries in the table (Figure \ref{fig:chartioVisualSQL}) \cite{chartioVisualSQL}. 

In the systems referred above the columns are added one by one sequentially, but other systems have different methods to select the table’s columns. For example, in Tableau Prep \cite{tableauPrep} and Microsoft Power BI \cite{powerBI} the table is chosen using a list, and all its attributes are added automatically (Figure \ref{fig:tableauPrep}). In these approaches, the users can remove, if they want, the columns not desired after. \cite{tableauPrepHelpWhatsNew} \cite{powerBIShapeAndCombineData}

Other systems, as Devart dbForge Query Builder \cite{dbForgeQueryBuilder} uses a diagram-based interface to select the entities and attributes of the query. In this system, the user can drag and drop the desired tables to the diagram area, and select through checkboxes the attributes needed, that are presented in the database schema diagram (Figure \ref{fig:devartDbForge}). Figure \ref{fig:approaches_select_data_sources} presents some examples of the techniques referred above to select the entities and attributes of the query.


\begin{figure}[htbp]
    \centering
    \subcaptionbox{Chartio Data Explorer\label{fig:chartioDataExplorer} \cite{chartioDataExplorer}}%
      {\includegraphics[width=0.5\linewidth]{chartio-data-explorer-data-source}}%
    \subcaptionbox{Chartio Visual SQL\label{fig:chartioVisualSQL} \cite{chartioVisualSQL}}%
      {\includegraphics[width=0.5\linewidth]{chartio-visual-sql-data-source}}%
      \\
    \subcaptionbox{Tableau Prep\label{fig:tableauPrep} \cite{tableauPrepHelpWhatsNew}}%
    {\includegraphics[width=0.5\linewidth]{tableau-data-source}}%
  \subcaptionbox{Devart dbForge Query Builder\label{fig:devartDbForge} \cite{dbForgeQueryBuilder}}%
    {\includegraphics[width=0.5\linewidth]{dbforge-data-source}}%
  \caption{Different approaches to select the entities of the query.}
    \label{fig:approaches_select_data_sources}
  \end{figure}

  \bigskip

  \textbf{Merge Type:}



  When data of multiple tables are required, it is necessary to establish how to merge them. Then, the visual language needs to adopt an interaction and representation technique to specify it. When data of multiple tables are required, it is necessary to establish how to merge them. Then, the visual language needs to adopt an interaction and representation technique to specify it. To define a join in Devart dbForge Query Builder \cite{dbForgeQueryBuilder}, the user can only select the attributes’ checkboxes of the different tables and the system generates an inner join automatically. To specify the join kind in this system there is a button on the toolbar to select all rows of one table, another, or both. This option can be used to perform left, right and outer joins \cite{dbForgeMakingJoinsBetweenTables}.
  
  Another approach used by some systems, such as Chartio Data Explorer \cite{chartioDataExplorer} and Microsoft Power BI \cite{powerBI}, is a form-based interface to insert a join. In the former, two queries can be merged clicking on a button to popup a form which can be used to select the merge type and the first columns that will be merged using dropdowns \cite{chartioDataExplorer} (Figure \ref{fig:chartioDataExplorerJoin}). Also, if there are null values on the merge related columns, there is an option to include or not the null values match rows \cite{chartioJoiningDataAcrossDatabases}. Similarly, the latter provides a button to merge queries that opens a modal where can be chosen the attributes that will be used on the merge (viewing also a table preview) and the join kind, using a dropdown \cite{powerBIShapeAndCombineData} (Figure \ref{fig:powerBIJoin}).
  
  Tableau Prep \cite{tableauPrep} provides two options to start a join between two tables: clicking on a "add join" hover button above the table visual representation with suggestion of related tables, or merge the visual representation of two tables using a drag and drop action. After this selection, the inner join type is selected automatically by the system according to the tables relationship. \cite{tableauAggregateJoinOrUnionData} \cite{tableauAddMoreDataInTheInputStep} However, the user can configure the join in a dedicated section (Figure \ref{fig:tableauJoinPanel}) where it is possible to define the join type using a Venn Diagram, to manage the join clauses using dropdown lists to select the fields, including also some join clause recommendations based on the database schema. Moreover, a summary of the join result that contains counters with the values included and excluded by each table, in a visual way using diagrams is provided. Finally, there is presented a list of the values included and excluded, where the red values represent the values excluded, as well as a preview of the join result \cite{tableauAggregateJoinOrUnionData}.

  \begin{figure}[htbp]
    \centering
    \subcaptionbox{Chartio Data Explorer\label{fig:chartioDataExplorerJoin} \cite{chartioDataExplorer}}%
      {\includegraphics[width=0.4\linewidth]{chartio-data-explorer-join}}%
    \subcaptionbox{Microsoft Power BI\label{fig:powerBIJoin} \cite{powerBIShapeAndCombineData}}%
    {\includegraphics[width=0.4\linewidth]{power-bi-join}}%
    \\
    \subcaptionbox{Tableau Prep\label{fig:tableauJoinPanel} \cite{tableauAggregateJoinOrUnionData}}%
      {\includegraphics[width=0.8\linewidth]{tableau-join-panel}}%
  \caption{Data merging approaches.}
    \label{fig:approaches_select_data_sources}
  \end{figure}

\section{Data Visualization}
\label{sec:data_visualization}

\section{Data User Exprerience and Expressiveness}
\label{sec:data_user_experience_and_expressiveness}

\section{Discussion}
\label{sec:discussion}