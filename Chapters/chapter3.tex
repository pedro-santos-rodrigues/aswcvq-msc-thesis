%!TEX root = ../template.tex
%%%%%%%%%%%%%%%%%%%%%%%%%%%%%%%%%%%%%%%%%%%%%%%%%%%%%%%%%%%%%%%%%%%%
%% chapter3.tex
%% NOVA thesis document file
%%
%% Chapter with a short laext tutorial and examples
%%%%%%%%%%%%%%%%%%%%%%%%%%%%%%%%%%%%%%%%%%%%%%%%%%%%%%%%%%%%%%%%%%%%
\chapter{Proposed Solution}
\label{cha:proposed_solution}
This chapter presents the solution proposed to the problems presented, that 
includes a description of the process realized to understand why the people 
use SQL to made queries instead of Aggregates, such as Personal Interviews and 
a Quantitative Analysis of the queries which customers ran on the cloud. 
Furthermore, this chapter presents what is the scope of the project, so it will 
be explained what problems will be tacked in detail.

\section{Requirements Analysis}
\label{sec:requirements_analysis}
As referred above, this section presents the results of the analysis made to 
understand why developers use SQL to make queries instead of Aggregates through 
its Visual Querying Features.

\section{Proposed Implementation}
\label{sec:proposed_implementation}
Following, will be detailed the pretended project, indicating all the problems 
identified and all the approached that will be adopted.

\section{Scope Definition}
\label{sec:scope_definition}
In this section, will be presented what was the decisions made of what problems 
will be addressed in this thesis, once the initial problem presentation had a wide 
scope and it was concluded that it's not possible to resolve all the Aggregates 
expressiveness and experience problems in this project.