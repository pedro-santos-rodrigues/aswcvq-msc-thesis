%!TEX root = ../template.tex
%%%%%%%%%%%%%%%%%%%%%%%%%%%%%%%%%%%%%%%%%%%%%%%%%%%%%%%%%%%%%%%%%%%%
%% chapter2.tex
%% NOVA thesis document file
%%
%% Chapter with the template manual
%%%%%%%%%%%%%%%%%%%%%%%%%%%%%%%%%%%%%%%%%%%%%%%%%%%%%%%%%%%%%%%%%%%%
\chapter{Related Work}
\label{cha:related_work}
In this chapter, will be described and compared a set of technologies and techniques, which could be useful to explore solutions, and understand different points of view to manage problems, regarding the project scope. As the project is focused on the improvement of an interface that combines the query design with the query output viewer, some relevant technologies, regarding the former will be compared, and, in a different section, the related work that exists for the latter shall be presented. Moreover, since the users are a fundamental piece of this study, will be presented some information about how is its relationship with data and what are its expectations to these query systems.

\section{Visual Query Formulation}
\label{sec:visual_query_formulation}

The first relevant topic to analyse is the visual representation approaches, which can be applied to design an user interface capable to build queries through a visual language. Catarci \textit{et al.} \cite{visualQuerySystemsForDatabases_aSurvey} presented an interesting classification according to the visual formalism which the interface is based on:

\begin{itemize}
    \item \textbf{Form-based}: 
    \item \textbf{Diagram-based}: 
    \item \textbf{Icon-based}: 
    \item \textbf{Hybrid}: 
\end{itemize}

\section{Data Visualization}
\label{sec:data_visualization}

\section{Data User Exprerience and Expressiveness}
\label{sec:data_user_experience_and_expressiveness}

\section{Discussion}
\label{sec:discussion}