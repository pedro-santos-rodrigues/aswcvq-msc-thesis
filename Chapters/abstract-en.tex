%!TEX root = ../template.tex
%%%%%%%%%%%%%%%%%%%%%%%%%%%%%%%%%%%%%%%%%%%%%%%%%%%%%%%%%%%%%%%%%%%%
%% abstrac-en.tex
%% NOVA thesis document file
%%
%% Abstract in English
%%%%%%%%%%%%%%%%%%%%%%%%%%%%%%%%%%%%%%%%%%%%%%%%%%%%%%%%%%%%%%%%%%%%
The Low-Code development has turned the software development industry, accelerating and streamlining the applications development and deployment, reducing also the background required. At the same time, individual or professional demands require information gathering of databases.

The OutSystems Low-Code Development Platform provides a visual tool, called Aggregates, to formulate data queries through interaction and manipulation of visual components. This user interface reduces the necessity to resort on textual languages, such as SQL, to build queries. These visual query systems turn the query design process more effective and efficient, reducing also the background knowledge required.

However, Aggregates not only do not support all the existing SQL functionalities but also have usability problems in the provided interface. Leading end users of the platform to use SQL to build their data queries.

Through an iterative design process, this dissertation includes the design, implementation and evaluation of prototypes with different fidelity levels. The aim is to create a user interface integrated in the OutSystems Platform which improves the effectiveness and efficiency of the query formulation process.


% Palavras-chave do resumo em Inglês
\begin{keywords}
Human-centered Design, Visual Query Interfaces, Low-Code Development, Human-computer Interaction, Iterative Design, Database Querying
\end{keywords} 
