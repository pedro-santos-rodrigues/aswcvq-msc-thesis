%!TEX root = ../template.tex
%%%%%%%%%%%%%%%%%%%%%%%%%%%%%%%%%%%%%%%%%%%%%%%%%%%%%%%%%%%%%%%%%%%%
%% abstrac-en.tex
%% NOVA thesis document file
%%
%% Abstract in English
%%%%%%%%%%%%%%%%%%%%%%%%%%%%%%%%%%%%%%%%%%%%%%%%%%%%%%%%%%%%%%%%%%%%
This dissertation addresses a usability improvement of a graphical user interface that allows query formulation without using textual query languages, such as SQL. This visual tool, called Aggregates, is provided on the OutSystems Low-Code Development Platform, to formulate data queries, through interaction and manipulation of visual components. 

Since Aggregates do not support all the existing functionalities of SQL, the OutSystems Platform allows users to build queries using this textual query language. Nonetheless, by evaluating customers' SQL queries, it was revealed that a considerable subset of the queries written in SQL could have been formulated using the visual tool.

The users' interviews and the results of the SQL queries evaluation have foreseen that the cause of the reduced acceptance of the visual approach, could be the existing usability problems on the interface. Furthermore, the interface is inadequate to perform more complex queries, which involve more entities and conditions.

Through an iterative design process, this dissertation includes the design, implementation, and evaluation of prototypes with different fidelity levels. The aim is to optimize the effectiveness and efficiency of the process where the user communicates to the system what data they intend to extract from the database. Moreover, the readability and comprehension improvement of the query visual representation is intended, reducing the time and the effort required to understand what data will be gathering from the database. The final goal is a functional prototype incorporated on the OutSystems Platform which accelerates the query formulation process without harming the learnability of the system.


% Palavras-chave do resumo em Inglês
\begin{keywords}
Visual Query Interfaces, Low-Code Development, User-Centered Design, Human-computer Interaction, Iterative Design, Database Querying
\end{keywords} 
