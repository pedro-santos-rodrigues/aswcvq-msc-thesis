%!TEX root = ../template.tex
%%%%%%%%%%%%%%%%%%%%%%%%%%%%%%%%%%%%%%%%%%%%%%%%%%%%%%%%%%%%%%%%%%%%
%% abstrac-pt.tex
%% NOVA thesis document file
%%
%% Abstract in Portuguese
%%%%%%%%%%%%%%%%%%%%%%%%%%%%%%%%%%%%%%%%%%%%%%%%%%%%%%%%%%%%%%%%%%%%
O Desenvolvimento \textit{Low-Code} tem-se expandido nos últimos anos por revolucionar o processo de desenvolvimento de \textit{software}, acelerando-o e reduzindo o pré-requisito de conhecimentos técnicos necessários. Paralelamente, a consulta de dados a bases de dados apresenta-se cada vez mais como uma necessidade a nível pessoal e profissional na maioria dos setores, independentemente do conhecimento existente em sistemas de computadores.

A Plataforma de Desenvolvimento Low-Code \textit{OutSystems}, inclui no seu sistema os \textit{Aggregates}, que permitem que as consultas de dados sejam construídas através de uma interface gráfica, sem recorrer a liguagens textuais como o \textit{SQL}. Estes sistemas visuais de consulta de dados, tornam as consultas mais eficazes e eficientes, reduzindo também o nível de conhecimentos prévios necessários para a sua construção.

Acontece que, para além de este componente de consulta não suportar todas as funcionalidades existentes em \textit{SQL}, também apresenta problemas de usabilidade na interface gráfica disponibilizada. Levando utilizadores finais da plataforma a recorrer a \textit{SQL} para construir as suas consultas de dados.

Através de um processo de desenho iterativo, esta dissertação contempla o desenho, implementação e avaliação de protótipos com diferentes níveis de fidelidade com o objetivo de obter uma interface integraga na Plataforma OutSystems, que melhore a eficiência e eficácia do processo de construção de consultas de dados.

%Deste modo, os utilizadores do sistema e as suas respectivas necessidades serão analisados com vista a desenhar e implementar uma interface gráfica que melhore a eficiência e eficácia do processo de construção de consultas a bases de dados. Através de um processo de desenho iterativo, é também contemplado o desenho, implementação e avaliação de protótipos com diferentes níveis de fidelidade com o objetivo de obter uma interface integrada na Plataforma OutSystems.









% Palavras-chave do resumo em Português
\begin{keywords}
Desenho Centrado no Utilizador, Interfaces Gráficas de Consulta de Dados, Desenvolvimento \textit{Low-Code}, Interação Pessoa-Máquina, Desenho Iterativo, Consulta de Bases de Dados
\end{keywords}
% to add an extra black line
