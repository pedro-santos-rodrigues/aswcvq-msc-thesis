%!TEX root = ../template.tex
%%%%%%%%%%%%%%%%%%%%%%%%%%%%%%%%%%%%%%%%%%%%%%%%%%%%%%%%%%%%%%%%%%%
%% chapter1.tex
%% NOVA thesis document file
%%
%% Chapter with introduciton
%%%%%%%%%%%%%%%%%%%%%%%%%%%%%%%%%%%%%%%%%%%%%%%%%%%%%%%%%%%%%%%%%%%
\newcommand{\novathesis}{\emph{novathesis}}
\newcommand{\novathesisclass}{\texttt{novathesis.cls}}


\chapter{Introduction}
\label{cha:introduction}
Nowadays, database queries are required not only in computer systems areas but also in most sectors of professional environments or personal demands. The database information gathering claim needs should resort to actual technologies to optimize the time spent and reduce the errors of this query process since the most important is to obtain the intended information.

In the decades of the 1960s, \glspl{DBMS} arose, and later in 1970s new management systems that use relational models, designated as \glspl{RDBMS}. Besides, the first \glspl{DQL} appeared, such as \gls{SQL} \cite{sequel_aStructuredEnglishQueryLanguage} which was considered by the \gls{ANSI} and \gls{ISO} as the standard query language \cite{databaseManagementSystems}. Even though these new technologies provide a structured way to access databases, knowledge of relational logic and \gls{DQL} was mandatory to fetch data from relational databases. Thereby, only a subset of people could use these powerful querying technologies.

\Glspl{VQS} were defined by Catarci \textit{et. al.} \cite{visualQuerySystemsForDatabases_aSurvey} as “systems for querying databases that use a visual representation to depict the domain of interest and express related requests”. These systems use different visual representations and interaction strategies to build database queries. This visual approach could improve the user's learning curve and reduce the mandatory previous knowledge of a \gls{DQL}, which are more difficult to learn mainly for people without programming base knowledge. Furthermore, some visual interface mechanisms such as automatisms, accelerators, or feedback messages, can be explored in order to accelerate the querying formulation process and reduce the errors that users can make while they are building queries.

In that way, those visual systems are not only useful to users not familiarized with \glspl{DQL}. Conversely, some studies have revealed that visual languages might be convenient to the expert users too. For instance, the comparison made by Catarci and Santucci \cite{diagrammaticVsTextualQueryLanguages_aComparativeExperiment} concludes that diagrammatic languages can reduce the error rate of the queries, in comparison with the ones that were build using textual languages, even when they were formulated by expert users. These results have demonstrated that even expert users make mistakes in simple queries (e.g., they may not remember the name of the tables or the precise syntax of some language expressions). Therefore, the \glspl{VQI} should be built strategically in order to take advantage of their peculiarities that could optimize the querying process not only to the users with a low database querying experience level but also for highly experienced users.

 %It could be not necessary...
 %Nonetheless, the wide users’ background and the diversity of the data domain made that a lot of these systems need to be modeled to a specific domain. So, it is very difficult to find a global integrated solution that covers the necessities of all users on all domains being this personal or professional.

\section{Motivation}
\label{sec:motivation}

Low-Code Development is a recent development paradigm that seeks to reduce the time and effort spent on tasks that would not have a significant impact on the final product outcome. As high-level languages, APIs and third-party infrastructures have allowed developers to be more productive and focus on the most valued sections of the software they produce. Low-code approaches have followed this endeavor, using visual \glspl{IDE}, connectors between components and lifecycle managers to employ an abstraction layer on the high-level languages, removing concerns of infrastructure or pattern reimplementation. In that way, developers could focus on tasks that truly accelerate the growth of the end product, achieving the desired goals with greater efficiency \cite{outsystems_whatIsLowCode}.

The OutSystems provides a cloud solution of low-code development, which allows developers to build and deploy enterprise-grade applications though visual interactions optimizing the time, effort, and previous knowledge necessary. Then, the OutSystems Platform aims to provide an application development environment that could be used by users with different backgrounds to build, deploy and manage their applications, using good practices and state-of-the-art technologies, even if they do not need to concern about that. Therefore, the vision and potential of the OutSystems Platform are similar to the \glspl{VQI} under-mentioned above since both intend to facilitate, accelerate, and optimize processes through visual interaction.

In spite of development in OutSystems is based principally on visual languages, there is the possibility to use low-level code, written in textual languages, such as Java, .NET or \gls{SQL}, in order to increase the extensibility and the power of solutions.  In that way, there are alternatives to performing operations not supported by the low-code visual approaches. This is the difference between Low-Code and No-Code paradigms since in No-Code is not possible to use low-level code \cite{outsystems_lowcodeVsNocode}. Nonetheless, if the visual languages of low-code platforms are more robust, responding more thoroughly to users’ requirements, there is a diminished demand to resort to these textual programming languages which are high error-prone and have a worse learning curve, requiring also, on multiple situations, previous coding experience.

Furthermore, as mentioned by  Amaral \textit{et. al.} \cite{improvingTheDeveloperExperienceWithALowCodeProcessModellingLanguage}, web and mobile applications produced on OutSystems’ technology, have proven an increase in quality. Also, it was concluded that low-code developers are 10.9 times more productive than the standard of \gls{IT} Industry, which does not use these rapid software solutions \cite{outByNumbers2013}. These results reinforce the importance of the improvement of the visual languages used on these platforms.

Following that vision, the principal motivation is the existence of a platform component that accelerates the query building process and reduces the errors that could occur throughout, keeping the experience simple and understandable by all users.

\section{Problem Description}
\label{sec:problem_description}
The OutSystems Platform \cite{outsystemsPlatform} provides a \gls{VQL} that allows users to query data from databases through visual interactions. Using that interface, it is possible to perform some operations that are usually supported by textual \glspl{DQL}, namely join, filter, sort, and aggregation operations.

Although the existing interface turns the process of query formulation more simple and intuitive, it does not support all \gls{SQL} functionalities. Due to that lack of expressiveness, the platform allows its users to formulate queries using SQL. However, as \textit{Catarci et al.} referred \cite{diagrammaticVsTextualQueryLanguages_aComparativeExperiment}, visual languages could give advantages in query formulation for all users, including the ones that are proficient in SQL. Thus, it is important to provide a powerful and consistent interface in order to give users the possibility to accelerate the formulation process, reducing also the rate of errors that may arise. 
%Currently, the OutSystems’ solutions have been applied to digital transformation processes in multiple industries that deal with high quantities of data, in order to accelerate the application development processes, unlocking its value and growth.

The principal purpose of this visual interface is to provide a more visual and dynamic tool that accelerates the query formulation process and turns it easier and less error-prone. So, the existing interface should be a useful and efficient tool for developers due to the potential of that visual approaches above-mentioned. However, OutSystems knew there were developers that were not using the visual querying interface, maintaining their preference for SQL. Therefore, it was necessary to verify the main causes that lead users to not use that visual tool. 

At the beginning of this dissertation, the lack of functionalities (e.g., IN, NOT IN, EXISTS, NOT EXISTS, DISTINCT, UNION and the possibility to use subqueries) was indicated as a significant factor for users to use SQL. Nevertheless, the first interface explorations revealed impactful usability problems. That is an important point since it could considerably harm users' query formulation process, reducing the value proposition of the system which intends to accelerate the querying process, keeping it more effective and less error-prone. Moreover, there were metric results and user interviews that confirmed the presumption concluded after interface exploration, highlighting the user experience of the interface as the core subject to research and improve.

Beyond the motivation of turning the query building process faster, effective, and less error-prone, the following questions would guide the problem definition process in order to clearly understand the existing problems of the interface:

\begin{itemize}
  \item Why do OutSystems developers often use \gls{SQL} to formulate database queries?
  \item What are the main causes that users point out to use \gls{SQL}?
  \item Who are the users more unsatisfied with the current provided visual approach to retrieve data? What are their reasons?
\end{itemize}

%MScPlan:
%What query features are supported by the existing OutSystems visual query language? Why do OutSystems developers often use \gls{SQL} to perform database queries? Why are some queries that can be built visually written through \gls{SQL} instead? What are the main causes that users point out to use \gls{SQL}? Who are the users more unsatisfied with the current provided visual approach to retrieve data? What are their reasons?


Under the above-mentioned circumstances, the goal of this thesis is the design, implementation. and evaluation of a new and more powerful \gls{VQI} to provide an improved \gls{UX} that: 

\begin{itemize}
  \item Accelerates the query formulation process;
  \item Improves the readability of queries (i.e., turns easier to understand what data will be fetched from database);
  \item Maintains the interface simple and intuitive for all users even the ones that do not know SQL or OutSystems, reducing also the existing learnability curve, whenever possible.
\end{itemize}


\section{Research Questions}
\label{sec:research_questions}
The main research question that is being addressed in this dissertation is: 
%MScPlan: \textit{Can we enable OutSystems developers to easily do complex database queries without ever using \gls{SQL}?}

\begin{center}
  Can we enable OutSystems developers to easily do complex \\ database queries without ever using \gls{SQL}?
\end{center}

Regarding the main question and the diverse background of the system target users, it is important to research how can be developed a solution that covers the requirements of all user types with the best usability possible.

The following research questions focus on this usability trade-off which depends on the users and the system particularities:

\medskip

\textbf{Research Question 1:} Does the current implementation have usability problems for the less experienced users?

\medskip

Considering the \gls{VQI} already implemented, the most complex queries have not been correctly covered by the tool. However, it is important to analyze also if novice users had similar problems or others that could have an impact on the task performing.

\medskip

\textbf{Research Question 2:} Can experienced users take advantage in using the visual interface to build queries instead of \gls{SQL}?

\medskip

Since the users more experienced who know other textual query languages, such as \gls{SQL}, can use \gls{SQL} to perform the queries, are advantages for them in the usage of a \gls{VQI}? What are the advantages and disadvantages of this type of approach for these users?

\medskip

\textbf{Research Question 3:} Can expert users' \gls{UX} be improved without reducing the system's learnability and satisfaction for less experienced users?


\medskip

The usability attributes trade-off depends on the system's target users' expectations and requirements. Thus it is important to take into account if the development to improve the efficiency, effectiveness, and satisfaction of expert users does not harm the effectiveness and the learnability of the operations performed for the less experienced users.

\section{Main Expected Contributions}
\label{sec:main_exp_contributions}
This work aims to provide a set of contributions not only as scientific research but also as additional value to OutSystems. Thus, a summary of the main expected contributions for this project are presented below:

\begin{itemize}
  \item A synthesization of the design concepts, including relevant interaction and conceptual models, usability definitions, guidelines and principles, and a description of the processes to evaluate the \gls{HCI} in the context of a software analysis;
  \item Provide a state-of-the-art about what are the most significant \glspl{VQS}, presenting also a comparison of visual representation techniques and interaction strategies used, as well as other important features proper for this study;
  \item A description of the analysis made to identify what are the most impactful problems regarding the existing \gls{VQI}, which includes user interviews and data analysis;
  \item Design and implementation of a new graphical user interface prototype that tries to improve the existing solution to visually build queries. This prototype expects to fix some predominant problems selected as the most relevant to solve;
  \item An usability evaluation of the prototype developed through the use of user tests;
  \item Increase the number of users that prefer using the visual query interface instead of SQL.
\end{itemize} 

\section{Document Structure}
\label{sec:document_structure}

The remaining chapters of this thesis are organized as follows:

\begin{itemize}
  \item Chapter 2 - \nameref{cha:background}: introduces some design and usability concepts to be used in this work. Besides, it is provided a context of the OutSystems Platform current progress, which explains the functionalities of the existing data querying tool;
  \item Chapter 3 - \nameref{cha:related_work}: analyses the users' interaction with database systems to improve the interfaces' suitability to the target users of the system. Also, approaches used by other systems to create interfaces that allow to visually build queries, are described and compared, detailing the interaction strategies used.
  \item Chapter 4 - \nameref{cha:proposed_solution}: explains the work made in the requirements analysis, including user's interviews and metrics gathering, and the decisions of what are the principal and impactful problems to be tackled. Moreover, the design approach and process adopted to develop the solution will be detailed.
\end{itemize}
