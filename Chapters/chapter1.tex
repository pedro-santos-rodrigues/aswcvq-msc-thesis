%!TEX root = ../template.tex
%%%%%%%%%%%%%%%%%%%%%%%%%%%%%%%%%%%%%%%%%%%%%%%%%%%%%%%%%%%%%%%%%%%
%% chapter1.tex
%% NOVA thesis document file
%%
%% Chapter with introduciton
%%%%%%%%%%%%%%%%%%%%%%%%%%%%%%%%%%%%%%%%%%%%%%%%%%%%%%%%%%%%%%%%%%%
\newcommand{\novathesis}{\emph{novathesis}}
\newcommand{\novathesisclass}{\texttt{novathesis.cls}}


\chapter{Introduction}
\label{cha:introduction}
This project was developed on a particular environment, so this chapter will 
introduce this thesis, starting by all the contextualization about the company, 
the product, and his section which will be the nuclear focus of this thesis, 
followed by describing the motivation behind it. In addition, will be presented 
an overview of the problem, such expected contributions and the structure of the 
document. 

\section{Context} 
\label{sec:context}
Once this project was proposed to improve the visual way that the users of the 
OutSystems Platform can query data, in this section, will be provided first some 
context about the OutSystems, in order to obtain an overview of the product. 
After this, also will be provided with a description of Aggregates, which is the 
current OutSystems feature to build queries through a visual interaction.

\subsection{OutSystems Context}
\label{subsec:outsystems_context}
In this section will be presented an overview of OutSystems Platform. The idea
here is to start describing first what are the main features of the product, and 
next to indicate briefly how is the impact of the product on the market.

\subsection{Aggregates Context}
\label{subsec:aggregates_context}
The main topic of this thesis is the expressiveness and the experience of 
Aggregates. Thus, in this section will be explained what is the Aggregates Feature, 
how can be used to provides a better experience to the users.

\section{Motivation}
\label{sec:motivation}
It will be presented the motivation behind this project. So, it will be explained 
what are the user possibilities to build queries, including visual and not 
visual ways, and what are the problems related with them. Also, will be 
presented what is the company motivation to do this project, since users can do 
the queries that they want in different ways.

\section{Problem Description}
\label{sec:problem_description}
In this section, will be presented generally the problem of this thesis. The goal 
of this section is to provide a global view of the problem without many details.

\section{Research Questions}
\label{sec:research_questions}
Here, will be enumerated the most important question which guides the workflow
(e.g. Can we enable OutSystems developers to easily do most kinds of database 
queries without ever using SQL?). The goal is to do all the research and 
development in order to obtain answers to all of these questions.

\section{Main Expected Contributions}
\label{sec:main_exp_contributions}
In the Main Expected Contributions, will be enumerated what are the thesis 
contributions to this work and to the proposed final solution.

\section{Structure}
\label{sec:structure}

The remaining chapters of this thesis are organized as follows:

\begin{itemize}
  \item Chapter 2 - \nameref{cha:related_work}: presents a short description
  of the OutSystems Platform, as well as a description of the existing techniques
  that already exist on the context of the main topic of this thesis - data visualization
  and visual querying. Besides, will be enumerated other commercial applications
  which can have relevant content for this study;
  \item Chapter 3 - \nameref{cha:proposed_solution}: describes the proposed solution,
  starting with a requirements analysis, followed by a more detailed explanation
  about the problem, and finally with a definition of the development scope to
  understand what problem will be tackled on detail;
  \item Chapter 4 - \nameref{cha:work_plan}: includes a planning
  of the total work inherent. Thus, will be presented an overview of the tasks that
  was done on this dissertation plan, together with the preview of the work which
  will be the focus of the second phase of the thesis, the elaboration.
\end{itemize}
