%!TEX root = ../template.tex
%%%%%%%%%%%%%%%%%%%%%%%%%%%%%%%%%%%%%%%%%%%%%%%%%%%%%%%%%%%%%%%%%%%
%% chapter1.tex
%% NOVA thesis document file
%%
%% Chapter with introduciton
%%%%%%%%%%%%%%%%%%%%%%%%%%%%%%%%%%%%%%%%%%%%%%%%%%%%%%%%%%%%%%%%%%%
\newcommand{\novathesis}{\emph{novathesis}}
\newcommand{\novathesisclass}{\texttt{novathesis.cls}}


\chapter{Introduction}
\label{cha:introduction}
Nowadays, database queries are required not only in computer systems areas but also in most sectors of professional environments or personal demands. This database information gathering claim needs should resort to actual technologies to optimize the time spent and reduce the errors of this query process since the most important is the obtaining of the intended information.

In the decades of the 1960s, \glspl{DBMS} arose, and later at 1970s new management systems that use relational models, designated as \glspl{RDBMS}. Moreover, the first \glspl{DQL} appeared, like \gls{SQL} \cite{sequel_aStructuredEnglishQueryLanguage} which was considered by the \gls{ANSI} and \gls{ISO} as the standard query language \cite{databaseManagementSystems}. These technological evolutions have improved the effectiveness and efficiency of the querying process. However, to find more specific and complex information on databases a higher degree of \gls{DQL} understanding was required. Thus, if from one side, the technological evolution and the digital transformation have optimized the data querying process, only a subset of people could use these powerful querying technologies.

\Glspl{VQS}, defined by Catarci, \textit{et. al.} \cite{visualQuerySystemsForDatabases_aSurvey} as “systems for querying databases that use a visual representation to depict the domain of interest and express related requests”, are used to mitigate some problems already referred. These systems use different visual representations and interaction strategies to make database queries, using a more intuitive visual approach, instead of using textual languages, which are more difficult to learn mainly for people without programming base knowledge.

Visual Query Systems (VQSs), defined by Catarci, \textit{et. al.} \cite{visualQuerySystemsForDatabases_aSurvey} as “systems for querying databases that use a visual representation to depict the domain of interest and express related requests”, are used to mitigate some problems already referred. These systems use different visual representations and interaction strategies to make database queries, using a more intuitive visual approach, instead of using textual languages, which are more difficult to learn mainly for people without programming base knowledge. Besides, even if it is not mandatory to be considered a \gls{VQS}, some systems have also data visualization features which can be useful to view the query result, or even the possibility to manage the database schema visually.

However, usually, these visual languages are associated as more useful to naive users, while textual languages are associated with more expert users.  Conversely, some studies have revealed that visual languages might be convenient to the expert users too. For example, the comparison made by Catarci and Santucci \cite{diagrammaticVsTextualQueryLanguages_aComparativeExperiment} concludes that diagrammatic languages can reduce the error rate of the queries, in comparison with those made in a textual language by expert users. These results imply that even expert users make mistakes in simple queries (e.g., they may not remember the name of the tables or the precise syntax of some language expressions). Thus, it is important to analyze how those languages could be used to optimize the querying process not only to the users with a low experience level but also for highly experienced users.

 %It could be not necessary...
 %Nonetheless, the wide users’ background and the diversity of the data domain made that a lot of these systems need to be modeled to a specific domain. So, it is very difficult to find a global integrated solution that covers the necessities of all users on all domains being this personal or professional.

\section{Motivation}
\label{sec:motivation}

Low-Code Development is a recent development paradigm that seeks to reduce the time and effort spent on tasks that will not have a significant impact on the final product outcome. Just as high-level languages, APIs and third-party infrastructures allowed developers to be more productive and focus on the most valued sections of the software they produce. Low-code approaches follow this endeavor, using visual \glspl{IDE}, connectors between components and lifecycle managers to employ an abstraction layer on the high-level languages, removing concerns of infrastructure or pattern reimplementation. In this way, one can focus on the tasks that truly accelerate the growth of the end product, achieving the desired goals with greater efficiency \cite{outsystems_whatIsLowCode}.

The goal of this dissertation is to analyze and improve the OutSystems Platform \cite{outsystemsPlatform} data querying component to create relational database queries through drag and drop interactions and simple configurations beyond visual components and multiple drag and drop features. The platform aims to provide the application development environment where users with different development backgrounds can build, deploy and manage their applications, following good practices and using state-of-the-art technologies.

However, conversely to No-Code approaches, the development through low-code systems gives, many times, the possibility to use low-level code, written in textual languages, such as Java, .NET or \gls{SQL}, in order to increase the extensibility and the power of low-code solutions \cite{outsystems_lowcodeVsNocode}. In that way, an alternative to perform the requests that are not supported by the low-code visual approaches is also provided. Nonetheless, if the visual languages of low-code platforms are more robust, responding more thoroughly to users’ requirements, there is a diminished demand to resort to these textual programming languages which are high error-prone and have a worse learning curve, requiring also, on multiple situations, previous coding experience.

Furthermore, as mentioned by  Amaral \textit{et. al.} \cite{improvingTheDeveloperExperienceWithALowCodeProcessModellingLanguage}, web and mobile applications produced on OutSystems’ technology, have proven an increase in quality. Also, it was concluded that low-code developers are 10.9 times more productive than the standard of \gls{IT} Industry, which does not use these rapid software solutions \cite{outByNumbers2013}. These results reinforce the importance of the improvement of the visual languages used on these platforms.


\section{Problem Description}
\label{sec:problem_description}
The OutSystems platform provides a \gls{VQL} that allows users to retrieve data from databases by simple processes. Besides, with this language, it is possible to perform some operations that are usually supported by textual \glspl{DQL}, namely join, filter, sort, and group operations.

Currently, the OutSystems’ solutions have been applied to digital transformation processes in multiple industries that deal with high quantities of data, in order to accelerate the application development processes, unlocking its value and growth.

To this extent, the already implemented querying tool is not able to deal with a set of scenarios, because it might not be accurate when the domain has a lot of tables involved or does not support essential advanced constructors. The following \gls{SQL} functionalities are an example of the not supported operations: IN, NOT IN, EXISTS, NOT EXISTS, DISTINCT, UNION and the possibility to use subqueries.

Accordingly, the following questions were fundamental to analyze and explore the problem and its requirements: 

\begin{itemize}
  \item What query features are supported by the existing OutSystems visual query language?
  \item Why do OutSystems developers often use \gls{SQL} to perform database queries?
  \item Why are some queries that can be built visually written through \gls{SQL} instead?
  \item What are the main causes that users point out to use \gls{SQL}?
  \item Who are the users more unsatisfied with the current provided visual approach to retrieve data? What are their reasons?
\end{itemize}

%MScPlan:
%What query features are supported by the existing OutSystems visual query language? Why do OutSystems developers often use \gls{SQL} to perform database queries? Why are some queries that can be built visually written through \gls{SQL} instead? What are the main causes that users point out to use \gls{SQL}? Who are the users more unsatisfied with the current provided visual approach to retrieve data? What are their reasons?


Under the above mentioned circumstances, the goal of this thesis is to design and evaluate a new and more powerful \gls{VQI} to provide an improved \gls{UX} that allows developers to formulate complex data queries intuitively and efficiently, without using \gls{SQL}.

\section{Research Questions}
\label{sec:research_questions}
The main research question that is being addressed in this dissertation is: 
%MScPlan: \textit{Can we enable OutSystems developers to easily do complex database queries without ever using \gls{SQL}?}

\begin{center}
  Can we enable OutSystems developers to easily do complex \\ database queries without ever using \gls{SQL}?
\end{center}

Regarding the main question and the diverse background of the system target users, it is important to research how can be developed a solution that covers the requirements of all user types with the best usability possible.

The following research questions focus on this usability trade-off which depends on the users and the system particularities:

\medskip

\textbf{Research Question 1:} Does the current implementation have usability problems for the less experienced users?

\medskip

Considering the \gls{VQI} already implemented, the most complex queries have not been correctly covered by the tool. However, it is important to analyze also if novice users had similar problems or others that could have an impact on the task performing.

\medskip

\textbf{Research Question 2:} Can experienced users take advantages in using the visual interface to build queries instead of \gls{SQL}?

\medskip

Since the users more experienced who know other textual query languages, such as \gls{SQL}, can use \gls{SQL} to perform the queries, are advantages for them in the usage of a \gls{VQI}? What are the advantages and disadvantages of this type of approach for these users?

\medskip

\textbf{Research Question 3:} Can expert users' \gls{UX} be improved without reducing the system's learnability and satisfaction for less experienced users?


\medskip

The usability attributes trade-off depends on the system's target users' expectations and requirements. Thus it is important to take into account if the development to improve the efficiency, effectiveness, and satisfaction of expert users does not harm the effectiveness and the learnability of the operations performed for the less experienced users.

\section{Main Expected Contributions}
\label{sec:main_exp_contributions}
This work aims to provide a set of contributions not only as scientific research but also as additional value to OutSystems. Thus, a summary of the main expected contributions for this project are presented below:

\begin{itemize}
  \item Increase the number of users that prefer using the visual query interface instead of SQL.
\end{itemize} 

\section{Document Structure}
\label{sec:document_structure}

The remaining chapters of this thesis are organized as follows:

\begin{itemize}
  \item Chapter 2 - \nameref{cha:background}: introduces some design and usability concepts to be used in this work. Besides, it is provided a context of the OutSystems Platform current progress, which explains the functionalities of the existing data querying tool;
  \item Chapter 3 - \nameref{cha:related_work}: analyses the users' interaction with database systems to improve the interfaces' suitability to the target users of the system. Also, approaches used by other systems to create interfaces that allow to visually build queries, are described and compared, detailing the interaction strategies used.
  \item Chapter 4 - \nameref{cha:proposed_solution}: explains the work made in the requirements analysis, including user's interviews and metrics gathering, and the decisions of what are the principal and impactful problems to be tackled. Moreover, the design approach and process adopted to develop the solution will be detailed.
\end{itemize}
