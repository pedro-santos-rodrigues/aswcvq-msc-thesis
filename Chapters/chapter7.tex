%!TEX root = ../template.tex
%%%%%%%%%%%%%%%%%%%%%%%%%%%%%%%%%%%%%%%%%%%%%%%%%%%%%%%%%%%%%%%%%%%
%% chapter1.tex
%% NOVA thesis document file
%%
%% Chapter with introduciton
%%%%%%%%%%%%%%%%%%%%%%%%%%%%%%%%%%%%%%%%%%%%%%%%%%%%%%%%%%%%%%%%%%%


\chapter{Conclusions and Future Work}
\label{cha:conclusions_and_future_work}

This chapter figures out a wide perspective and consideration of all points approached in this dissertation as well as the results achieved, reinforcing also what could be improved in the future.

\section{Conclusion}
\label{sec:conclusion}

The low-code development paradigm has accelerated software development and extended this industry to people with different backgrounds due to the usage of visual programming languages. Sharing the same vision, \glspl{VQI} arose in order to facilitate the data querying process turning it easier to lean, more efficient, and less error-prone.

In this dissertation, the \gls{VQI} integrated into the OutSystems low-code development platform was studied due to the low level of the users' acceptance and satisfaction of this platform component. Notwithstanding the lack of some \gls{SQL} features in the visual query building solution, a large set of usability problems were identified through users' interviews, quantitative analysis, issues reported in the OutSystems Community, and usability tests.

In a nutshell, the interface had different usability problems that were avoiding users to take advantage of the main advantages of a visual query language: easiness to learn, acceleration of the querying process, reduction of the errors which could occur during this process. 

Hence, all problems were detailed and categorized as the target users of the system were divided into groups according to their requirements and expectations in order to grant the solution built would have a positive effect on all users. Moreover, it was prepared a design and evaluation method that would make it possible to build the most appropriate solution to cover the problems identified and to improve the experience of the target users.

Firstly, a paper prototype was built to explore the first ideas and test with users in order to validate if the design choices made were forwarding the solution in the right direction to solve the problems users felt in the existing interface.

Secondly, the information obtained from the evaluation of the paper prototype was used to refine details and implement a new prototype integrated into the OutSystems Platform which aimed to mitigate some learnability problems in the interface and reducing the errors users could make while they are building their queries.

The existing interface and the final prototype were both tested with 30 users, using the same method and testing scenarios in order to compare the success of the design changes applied in the interface. The new interface 


\section{Future Work}
\label{sec:future_work}
