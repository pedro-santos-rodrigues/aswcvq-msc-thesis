%!TEX root = ../template.tex
%%%%%%%%%%%%%%%%%%%%%%%%%%%%%%%%%%%%%%%%%%%%%%%%%%%%%%%%%%%%%%%%%%%%
%% chapter3.tex
%% NOVA thesis document file
%%
%% Chapter with a short laext tutorial and examples
%%%%%%%%%%%%%%%%%%%%%%%%%%%%%%%%%%%%%%%%%%%%%%%%%%%%%%%%%%%%%%%%%%%%
\chapter{Proposed Solution}
\label{cha:proposed_solution}
After the problem contextualization, and the description of the related concepts, techniques and studies, this chapter presents the work done so far. Starting with a requirements analysis which includes user interviews and extraction of metrics to conclude in a quantitative way what are the cases where the SQL is used. Continuing with the priority assigned for each problem based on the requirement analysis. A current progress state of the solution developed will be presented, describing the actual development state and a foreseeing schedule of the work plan for the remaining time until the final of the project.

\section{Requirements Analysis}
\label{sec:requirements_analysis}
\textcolor{gray}{As referred above, this section presents the results of the analysis made to understand why developers use SQL to make queries instead of Aggregates through its Visual Querying Features.}

\section{Scope Definition}
\label{sec:scope_definition}
\textcolor{gray}{In this section, will be presented what were the decisions made of which problems will be addressed in this thesis, once the initial problem presentation had a wide scope and it was concluded that it's not possible to resolve all the Aggregates expressiveness and experience problems in this project.}

\section{Proposed Implementation}
\label{sec:proposed_implementation}
\textcolor{gray}{Following, will be detailed the intended project, indicating all the problems identified and all the approached that will be adopted.}

\section{Work Plan}
\label{sec:work_plan}

