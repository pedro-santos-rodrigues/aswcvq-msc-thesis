%!TEX root = ../template.tex
%%%%%%%%%%%%%%%%%%%%%%%%%%%%%%%%%%%%%%%%%%%%%%%%%%%%%%%%%%%%%%%%%%%%
%% chapter3.tex
%% NOVA thesis document file
%%
%% Chapter with a short laext tutorial and examples
%%%%%%%%%%%%%%%%%%%%%%%%%%%%%%%%%%%%%%%%%%%%%%%%%%%%%%%%%%%%%%%%%%%%
\chapter{Proposed Solution}
\label{cha:proposed_solution}
After the problem contextualization, and the description of the related concepts, techniques and studies, this chapter presents the work done so far. Starting with a requirements analysis which includes user interviews and extraction of metrics to conclude in a quantitative way what are the cases where the SQL is used. Continuing with the priority assigned for each problem based on the requirement analysis. A current progress state of the solution developed will be presented, describing the actual development state and a foreseeing schedule of the work plan for the remaining time until the final of the project.

\section{Requirements Analysis}
\label{sec:requirements_analysis}
The project has started with an exploration of the OutSystems Platform data querying tool, the Aggregates, to understand its functionalities and visual approaches used to perform queries visually, as has been described in section \ref{subsec:visual_data_querying}. Meanwhile, not only were identified the expressiveness problems of the visual language, previously mentioned in section \ref{sec:problem_description}, but also usability problems could be perceived in the performance of available actions. 

After that, user interviews have been prepared to explore better the barriers of the tool and understand the impact of the problems in user actions. So, in the interviews, after perceiving the background of the participant, more directed questions were asked to pick up the first responses of the users about the advantages and disadvantages of the visual query tool. Going straight to this point is expected that users talk about the more impactful problems for them and the main situations where this tool is useful. Next, it was asked what are the situations when they use SQL to obtain insights about the reason to not use the visual tool to perform these queries. Also, were required that users present examples to explain the problems they have, whenever possible, because this strategy can be useful to understand the impact level of the problem and sometimes show other problems not referred. In addition, when the users did not mention anything about expressiveness problems, direct questions about if they never needed to use these operations in Aggregates were colocated to cover the possibility of forgetting to mention that aspect.

The results reveal that the most novice users, with less than six months of experience, feels that Aggregates are easy to use and covers his necessity, referring also that is easier to learn than SQL. The most experienced users, who use the OutSystems platform to develop applications every day for professional purpose and have a technological background, reported that in the visual tool they cannot have a good understanding of the global view of the query, mainly if many tables, columns and business rules are involved. Switch between tabs in the interface to view the data sources, and the filtering and sorting criteria were other issue presented, as well as the lack of control on the query output. Asked about the aggregation functions, which were implemented when Simple Queries have been replaced by Aggregates, they do not refer any problem with the approach interaction strategy adopted, not considering these new features as a problem that blocks the use of the tool. Furthermore, other usability problems that decrease the users’ satisfaction have been pointed out, like the difficulty to search for a column in the query result, when there are many columns.



\section{Scope Definition}
\label{sec:scope_definition}
\textcolor{gray}{In this section, will be presented what were the decisions made of which problems will be addressed in this thesis, once the initial problem presentation had a wide scope and it was concluded that it's not possible to resolve all the Aggregates expressiveness and experience problems in this project.}

\section{Proposed Implementation}
\label{sec:proposed_implementation}
\textcolor{gray}{Following, will be detailed the intended project, indicating all the problems identified and all the approached that will be adopted.}

\section{Work Plan}
\label{sec:work_plan}

