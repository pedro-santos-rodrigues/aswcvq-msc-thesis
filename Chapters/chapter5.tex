%!TEX root = ../template.tex
%%%%%%%%%%%%%%%%%%%%%%%%%%%%%%%%%%%%%%%%%%%%%%%%%%%%%%%%%%%%%%%%%%%
%% chapter1.tex
%% NOVA thesis document file
%%
%% Chapter with introduciton
%%%%%%%%%%%%%%%%%%%%%%%%%%%%%%%%%%%%%%%%%%%%%%%%%%%%%%%%%%%%%%%%%%%


\chapter{Methodologies}
\label{cha:methodologies}
Taking into account the existing usability problems and the target users' requirements and expectations, a methodology is essential to design, implement, and evaluate solutions throughout the development process. Therefore, this chapter presents the approached phases and techniques across the solution conception, from the first sketches to the final evaluations. 

\section{Iterative Design}
\label{sec:iterative_design}
Before starting the solution building process, in order to solve the presented problems, an iterative design methodology was defined. Accordingly, it is presented the design strategy adopted, which will be further detailed in \nameref{cha:design_and_implementation}. This methodology uses an iterative design strategy in order to keep a user-centric design, which prioritizes the users' needs according to the mentioned in \ref{subsec:user_centered_design}. Regarding the prototyping method, the evolutionary prototyping principle (described in \ref{subsubsec:sketching_and_prototyping}) was applied, where the last prototype is used as a baseline to develop the prototype of the next iteration.

Therefore, the following phases are the ones included in the methodology used to build iteratively the solution which takes into account the users adoption at the final of each iteration:

Therefore, the following phases presented were used to iteratively build the solution which took into account the users adoption at the final of each iteration:

\begin{itemize}
    \item \textbf{Sketching: }The design process started with an initial sketching phase, where the first solution ideas were explored and crafted. This is a favorable technique to contemplate how new ideas could be integrated into the existing interface. The most important aspect is to think about system transversal changes and not about particular details of specific components, since these details can be refined later. The outcome of this phase should set a more concrete idea in what can be inserted in the prototype, even if it is necessary to think more about how to implement it later.
    \item \textbf{Paper Prototype: }
    \begin{itemize}
        \item \textbf{Design: }
        \item \textbf{Implementation: }
        \item \textbf{Evaluation: }
    \end{itemize}
    \item \textbf{Service Studio Implementation (Final Prototype): }
    \begin{itemize}
        \item \textbf{Design: }
        \item \textbf{Implementation: }
        \item \textbf{Evaluation: }
    \end{itemize}
\end{itemize}

\section{Testing Scenarios}
\label{sec:testing_scenarios}

\section{Evaluation Technique}
\label{sec:evaluation_technique}
