%%%%%%%%%%%%%%%%%%%%%%%%%%%%%%%%%%%%%%%%%%%%%%%%%%%%%%%%%%%%%%%%%%%%
%% appendix1.tex
%% NOVA thesis document file
%%
%% Chapter with example of appendix with a short dummy text
%%%%%%%%%%%%%%%%%%%%%%%%%%%%%%%%%%%%%%%%%%%%%%%%%%%%%%%%%%%%%%%%%%%%
\chapter{Taxonomy of Problems - Existing Interface}
\label{app:taxonomy_of_problems_existing_interface}

The visual interface to formulate queries that was implemented before this dissertation had several usability problems which led users to prefer to use other \glspl{DQL} such as SQL. In order to comprehend the existing problems of the visual interface or the characteristics that hamper users to extract advantages of that visual approach to query databases, there were performed the following studies:

\begin{itemize}
    \item \textbf{Study and Analysis:} This was the first approach used to explore the existing problems of the interface. The process started beyond the visualization of two OutSystems tutorials \cite{outsystems_tutorial_aggregates_101, outsystems_tutorial_advanced_aggregates} about visual data querying. This was the first experience using the interface and there were pointed out some problems. As this was the first experience the were found principally issues regarding hidden operations and behaviors that were not clear for users who didn't use to the platform. After that tutorials, there has been made some more explorations using practical examples where there were found other problems related to efficiency and effectiveness of use;
    \item \textbf{User Interviews:} There were performed some dialogues with a reduced set of novice and expert users to comprehend what are the main issues pointed out. In case of expert users, the causes to use SQL instead of the Visual Interface were registered as well as a set of other \gls{UX} issues. Regarding users who did not have relevant experience either in SQL or OutSystems, it was registered the functionalities more difficult to learn or understand;
    \item \textbf{Community Ideas:} Since OutSystems contains a wide and worldwide Community where users could contribute with suggestions or express their problems or difficulties, all the ideas of the category "Aggregates and Queries" were explored to understand the problems presented. Moreover, the number of likes of each post was registered in order to know what are the problems which had more user reactions; 
    \item \textbf{User Testing:} During the user tests of the existing implementation, there were pointed out some issues by users. Also, other problems were found through the interpretation of user-interface interaction.
\end{itemize}

Furthermore, each issue registered was characterized according to the Nielsen Heuristics \cite{nielsen_heuristics} and the artifact and task attributes of a Framework adapted from Usability-ODC Framework \cite{in_process_usability_problem_classification_analysis_improvement}. Besides, for each issue it was checked if the action most hampered is the query formulation or the query comprehension as well as what are the interface components affected between the ones presented below:

\begin{itemize}
    \item Actions and Nodes (Method/Function where the Visual Query was added)
    \item General Layout (The arrangement of the main interface components on the main window)
    \item Sources
    \item Joins
    \item Filters
    \item Sorting
    \item Aggregation Functions
    \item Calculated Attributes
    \item Static Entities
    \item Query Result Table
    \item Test Values
\end{itemize}

In that way, it is simple to categorize and prioritize the problems detected. Besides, as mentioned above, if the problem were referred in the OutSystems Community, there were included the respecting posts and their number of likes. Table X represents the result of all problems detected and categorized in accordance with the parameters mentioned:

