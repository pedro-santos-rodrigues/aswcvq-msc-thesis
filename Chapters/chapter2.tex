%!TEX root = ../template.tex
%%%%%%%%%%%%%%%%%%%%%%%%%%%%%%%%%%%%%%%%%%%%%%%%%%%%%%%%%%%%%%%%%%%%
%% chapter2.tex
%% NOVA thesis document file
%%
%% Chapter with the template manual
%%%%%%%%%%%%%%%%%%%%%%%%%%%%%%%%%%%%%%%%%%%%%%%%%%%%%%%%%%%%%%%%%%%%
\chapter{Related Work}
\label{cha:related_work}
On this thesis, it is pretended to apply Human-computer interaction (HCI), Data Visualization and Visual Querying concepts, techniques and technologies to improve a Visual Querying Feature of the OutSystems Low-code Development Platform. Thus, in this chapter, will be presented the results of a study that analysed what is the Low-code development platform background and its actual situation, as well as what are the techniques and technologies which already exist, including some comparison between them. Finally, will be enumerated what products, technologies and tools exist on other commercial applications which can be related with the topics of this thesis.

\textcolor{blue}{It is necessary to add more sections to describe some key concepts about user experience testing and analysis.}

\section{OutSystems Background}
\label{sec:outsystems_background}
The entirety of this thesis has the aim of improving the OutSystems Platform, so it is very important to understand what is that product and what can be developed with it.
\textcolor{gray}{This section provides an overview of this Low-code Development Platform, describing its value proposal, and its technological goals and approaches. In addition, it will be used some images to illustrate some relevant aspects of the platform.}

\textcolor{blue}{The next sections depend on the research done.}

\section{Data Visualization}
\label{sec:data_visualization}

\section{Visual Queries}
\label{sec:visual_queries}

\section{Data User Exprerience and Expressiveness}
\label{sec:data_user_experience_and_expressiveness}

\section{Technologies and Commercial Applications}
\label{sec:technologies_and_commercial_applications}
Such as the techniques research, it is also very important to search what are the technologies related with the subjects of this thesis that already exist, as this knowledge can be very important to the concept of a solution proposal.

Furthermore, in any research, the academic content should not be the only taken into account, because sometimes the knowledge does not evolve only in the research centres but also in the companies. Since this thesis is made to improve a company product, the latter assertion has additional strength, so also, will be introduced commercial applications which can be useful to all the entirety of this process.